\section{A Review On Deep Learning In UAV Remote Sensing}

%4
\begin{frame}
    \frametitle{\textit{A Review On Deep Learning In UAV Remote Sensing}
    ~~------~~ Applications}
    \begin{block}{Main task}
        A brief review of DL methods and their applications on the UAV-based imagery
        to solve classification, object detection, and semantic segmentation
        problems.
    \end{block}
    \begin{itemize}
        \item Environment Mapping
        \item Urban Mapping
        \item Agriculture Mapping
    \end{itemize}

\end{frame}

%1
\begin{frame}
    \frametitle{\textit{A Review On Deep Learning In UAV Remote Sensing}
    ~~------~~ Overview}

    \colorbox{orange}{Highlights:}
    \begin{itemize}
        \item UAVs offer flexibility in data collection, they present the
            dynamic data with the embedded RBG camera, multispectral,
            hyperspectral, thermal and LiDAR sensors.
        \item The sensor on the UAVs can generate different altitude and
            point-of-view data, which contains more features like patterns and
            textures etc.
        \item This survey is particularly related to UAV-imagery, \color{red}{
            while some of the others focus on the DL in remote sensing but not
            UAV-based imagery.}
    \end{itemize}

\end{frame}

%2
\begin{frame}
    \frametitle{\textit{A Review On Deep Learning In UAV Remote Sensing}
    ~~------~~ Deep Neural Networks}

    \begin{block}{DNN:}
        \begin{itemize}
            \item Models Types;
            \item Validation metrics: ROC curve, F-score, IoU and other metrics.
        \end{itemize}
    \end{block}

    \begin{itemize}
        \item \textbf{classification}\linebreak
            labeling the image or the patch according to their classes.
        \item \textbf{object detection}\linebreak
            draw a boundary box around the object and labeling them.
        \item \textbf{semantic and instance segmentation}\linebreak
            draw the region or structures around the boundary of object,
            i.e., distinguish the object at the pixel level.
            \begin{itemize}
                \item The semantic segmentation can not distinguish multiple
                    objects of the same category.
                \item detect objects in pixel-level mask and labeling each
                    mask into a class label.
            \end{itemize}
        \item \textbf{regression}\linebreak
            In comparison to classification, regression tasks using DL is
            not often used
    \end{itemize}

\end{frame}

%3
\begin{frame}
    \frametitle{\textit{A Review On Deep Learning In UAV Remote Sensing}
    ~~------~~ Approaches of DL}

    \begin{block}{Scene-Wise Classification, Object Detection, and Segmentation}
    \end{block}
    \begin{itemize}
        \item \textbf{One-stage detectors:}\linebreak
            regression-based methods, directly make classification and detect
            the location of the objects. They tend to have lower accuracy.
        \item \textbf{Two-stage detectors:}\linebreak
            region proposal-based methods, usually 2 steps: \linebreak
            \begin{itemize}
                \item generate the candidate boundary box(region)
                \item classify the regions into object classes, refine the
                    boundary.
            The widely used algorithm is the strategy proposed with
            \textbf{Faster-RCNN} and also other methods.
            \end{itemize}
    \end{itemize}

\end{frame}


%5
\begin{frame}
    \frametitle{\textit{A Review On Deep Learning In UAV Remote Sensing}
    ~~------~~ Problems to be settled}

    \begin{itemize}
        \item real-time implementation on the UAVs
        \item The accuracy decreases due to the noise from the hyperspectral or
            high-dimensional data.
        \item The difference between the training and testing images.
        \item domain adoption and transfer learning
        \item attention based mechanisms ~------~ transformer
        \item Multi tasks learning
    \end{itemize}

\end{frame}
