\section{A review of machine learning applications in wildfire science and
management}

%1
\begin{frame}
    \frametitle{\textit{A review of machine learning applications in wildfire
    science and management} ~~------~~ Overview}
    \begin{columns}[t]
        \column{0.45\textwidth}
        \begin{block}{Applications}
            The application on the wild fire can be sperated into 6 domians
            (the problem domain):
        \end{block}
        \begin{itemize}
            \item fuels characterization, \textbf{fire detection and mapping}
            \item fire weather and climate change
            \item fire occurrence and risk
            \item \textbf{fire behavior prediction}
            \item fire effects
            \item \textbf{fire management}
        \end{itemize}

        \column{0.45\textwidth}
        \begin{block}{ML Methods}
            The most often used ML model across all of the problem domains:
        \end{block}
        \begin{itemize}
            \item Traditional Methods:
                \begin{itemize}
                    \item random forest (RF)
                    \item[*] maximum entropy (MaxEnt)\footnotemark[1]
                    \item artificial neural network (ANN)
                    \item decision tree (DT)
                    \item support vector machine (SVM)
                    \item genetic algorithm (GA)
                \end{itemize}
            \item Modern Methods:
                \begin{itemize}
                    \item deep learning (including the CNN, LSTM and other object
                        detection methods)
                    \item agent-based learning (including the Renforcement
                        Learning)
                \end{itemize}
        \end{itemize}
    \end{columns}
    \footnotetext[1]{The item with * should be studied later if needed.}
\end{frame}

%2
\begin{frame}
    \frametitle{\textit{A review of machine learning applications in wildfire
    science and management} ~~------~~ Main Tasks}
    \begin{enumerate} 
        \item fuels characterization, \textbf{fire detection and mapping:}\\
            \begin{itemize}
                \item \textbf{Detection (Classification + Regression problem
                    $\rightarrow$ Object Detection):}

                    Automatically detecting the wildfires as soon as possible,
                    even it is realtively small, with heat signatures or smoke
                    in infra-red or optical images, which can extend  spatial
                    and temporal coverage of the detection.

                \item \textbf{Mapping (Classification):} 
                    Classify the burned and unburned area to get the perimeter
            \end{itemize}
        \item \textbf{fire behavior prediction (Regression )\footnotemark[1]}:

            The papers in this area mainly deal with the larger scale processes
            and characteristics to predict or estimate the fire spread rates,
            fire growth, burned area, and fire severity.

        \item \textbf{fire management (Optimization, Regression)}:

            To have the appropriate amount of fire on the land.
            \begin{itemize}
                \item Optimize the allocation of the people, watchtower or other
                    fire-managing elements.
                \item Estimate the relationship of the fire characteristics and
                    the wild fire responds and even the social factors.
            \end{itemize}
    \end{enumerate}
    \colorbox{orange}{However, may be we can do more further about these
    applications!}
    \footnotetext[1]{not sure.}

\end{frame}

%3
\begin{frame}
    \frametitle{\textit{A review of machine learning applications in wildfire
    science and management} ~~------~~ Proposed Theories and Methods}

    The input data could be images, infra-images, multispectral images and MOIDS
    hotpot data from the UAVs, manned aircraft, watchtower, and satellite
    \begin{table}[H]
        \centering
        \begin{tabular}{p{0.3\textwidth}p{0.3\textwidth}p{0.3\textwidth}}
            \hline
            \textbf{Application Type} & Traditional Methods                    & Modern Methods  \\
            \hline
            fire detection            & ANN+infra SVM+video GA+LiDAR ANFIS     & CNN LSTM YOLO Optica-Flow 3D-CNN    \\
            fire mapping              & RF ANN SVM GPR(GP regression) AdaBoost & DNN    \\
            fire behavior prediction  & RF BNs KNN GA+physical model MDP       & CNN LSTM    \\
            fire management           & GA(optimization) BN ANN                & $\backslash$\\
            \hline
        \end{tabular}
        \caption{proposed methods}
    \end{table}
    \begin{itemize}
        \item \textbf{Traditional Methods:} Usually extract the images' feature
            with the computer vision technology, for example, the
            color(may under differernt color sapce) , texture,
            feature descriptors, etc.
        \item \textbf{Modern Methods:} usually the end-to-end model, use the
            convolution layers to `extract features`, the features are
            usually incomprehensible.
    \item \textbf{Mixed Methods:} Fast-RCNN, Faster-RCNN+feature selecting based on
            fuzzy logic
    \end{itemize}

\end{frame}
